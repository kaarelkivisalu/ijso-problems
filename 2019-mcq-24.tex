\ylDisplay
{}% Problem name
{2019}% Year
{mcq}% Round (mcq, theory, experiment)
{24}% Problem nr.
{physics}% Subject (physics, chemistry, biology)
{}% Difficulty (1-3)
{
% Syl:
\ifStatement
A concave mirror of focal length $f=\SI{0.50}{\m}$ is placed on a base as shown in the figure. A ball of mass $m$ falls from a height of \SI{1.2}{\m} vertically along the principal axis towards the mirror. If the  ball loses \SI{16}{\percent} of its energy after each collision with the mirror,  what is the distance between the ball and the image formed by the  mirror when the ball reaches its maximum height after the second  collision? Assume that the falling ball does not break the mirror!
\begin{center}
  \includegraphics[width=0.5\linewidth]{2019-mcq-24-p}
\end{center}
\fi


\ifOption1
\SI{0.37}{\m}
\fi


\ifOption2
\SI{0.55}{\m}
\fi


\ifOption3
\SI{0.66}{\m}
\fi


\ifOption4
\SI{0.75}{\m}
\fi


\ifHint

\fi


\ifSolution

\fi


\ifEstStatement
% Problem name:

\fi


\ifEstOption1

\fi


\ifEstOption2

\fi


\ifEstOption3

\fi


\ifEstOption4

\fi


\ifEstHint

\fi


\ifEstSolution

\fi
}
