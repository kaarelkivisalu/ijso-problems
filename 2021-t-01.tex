\ylDisplay
{Bullet and Cannon}% Problem name
{2021}% Year
{theory}% Round (mcq, theory, experiment)
{1}% Problem nr.
{physics}% Subject (physics, chemistry, biology)
{}% Difficulty (1-3)
{
% Syl:
\ifStatement
\subpart[The Modern day bullet]
Nitroglycerin is one of the important ingredients in modern day bullets. The self-combustion of this material is written as

\ce{2C3H5N3O9 -> 6CO2 + 3N2 + 5H2O + 1/2O2 + \text{Head}}

The amount of heat released is \SI{666}{\kJ} for 2 mole of nitroglycerine.

\SI{11.35}{\g} of this material is used in a cartridge of single bullet. The mass of the actual bullet is \SI{100.0}{\g}.

\prob
Find the molar mass of nitroglycerine.

\prob
Find the number of moles of nitro-glycerine in one bullet cartridge.

\prob
Find the amount of energy released (numerical value in SI unit) during combustion of one bullet.

\prob
Assuming that the entire energy evolved during combustion is used to give kinetic energy to the bullet. Calculate the maximum possible muzzle speed (numerical value in SI unit) of this bullet.


\subpart[Traditional Cannon]

A traditional Cannon barrel of inner diameter \SI{15.0}{\cm} and length \SI{5.0}{\m} was filled with gunpowder (nitrocellulose) to 20\% of its length and topped with a cannon ball of same diameter as the barrel. Inner walls of the canon barrel are frictionless.

When it is fried, all of the nitrocellulose burns instantly and produces gas with pressure of 1000 standard atmosphere. When the ball exits the barrel the gas temperature drops to one third of the temperature (in K) at the time of ignition. Assume ideal gas situation. Neglect opposing atmospheric pressure.

\prob
Write the formula to find the pressure (final pressure $P_2$ in terms of initial pressure $P_1$, initial volume $V_1$, initial temperature $T_1$, final volume $V_2$ and final temperature $T_2$) when the cannon ball exits the barrel.

\prob
Calculate the pressure (numerical value in SI unit) on the ball when it exits the barrel. Express your answers in three significant figures i.e. two digits after decimal.

\prob
Calculate the force (numerical value in SI unit) on the ball when it exits the barrel. Express your answers in three significant figures i.e. two digits after decimal.


\fi


\ifOption1

\fi


\ifOption2

\fi


\ifOption3

\fi


\ifOption4

\fi


\ifHint

\fi


\ifSolution

\fi


\ifEstStatement
% Problem name:

\fi


\ifEstOption1

\fi


\ifEstOption2

\fi


\ifEstOption3

\fi


\ifEstOption4

\fi


\ifEstHint

\fi


\ifEstSolution

\fi
}
